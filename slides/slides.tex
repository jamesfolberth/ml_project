\documentclass{beamer}
\usetheme{Berlin}
\usepackage{amsmath}
\usepackage{lmodern}
\usepackage{hyperref}
\usepackage{color}

\usenavigationsymbolstemplate{}

\title[Science Questions - Milksteak]{Answering Questions about Science - Milksteak}
\subtitle[Application]{Quiz Bowl Science}
\author[Cowley, Folberth, Reiersen, Wiley]{Andrew Cowley, James Folberth, Derek Reiersen, Benjamin Wiley}
\institute[CU Boulder]{
  Final Project\\
  CSCI 5622: Machine Learning\\[1ex]
University of Colorado at Boulder
}
\date[December 2015]{December 15, 2015}

\begin{document}
\begin{frame}[plain]
  \titlepage
	\title[Science Questions]{Answering Questions about Science}
\end{frame}

%----------------------------------------------------------------------------------------------------------------------------------

% Our talk is short; don't want to spend time on an overview.
%\begin{frame}{Overview}
%	\tableofcontents
%\end{frame}


\begin{frame}{Problem Statement}
   \begin{itemize}
      \item We're given a bunch of multiple choice science questions, where each answer corresponds to a Wikipedia page title.\\
      \item The goal is to predict the correct answer to new questions.
   \end{itemize}

   Example: ``This phenomenon occurs twice as quickly in the Moran model as in the Wright-Fisher model.''
   \begin{enumerate}
      \item {\color{green} Genetic drift}
      \item Hamiltonian (quantum mechanics)
      \item Georg Wilhelm Friedrich Hegel
      \item Group (mathematics)
   \end{enumerate}
\end{frame}


%----------------------------------------------------------------------------------------------------------------------------------

\section{Methods}

\begin{frame}{Text-Based Logistic Regression Baseline}
   \begin{itemize}
      \item Use training question with correct answer as training data
      \item 62\% baseline
      \item Improve by using Wikipedia summary sentences as additional training data
      \item 72\% on cross-validation, 65 percent on Kaggle
   \end{itemize}
\end{frame}



%\subsection{A little NLP}
\begin{frame}{A little NLP}
   \begin{itemize}
      \item Download and cache Wiki summary, categories, full text, etc.
      \item Lemmatize question strings and Wiki summaries (shorter than full text).
      \item Try to emphasize proper nouns (e.g. {\color{blue} Moran model} of genetic drift).
      \item Count tokens and build feature vector.
   \end{itemize}
\end{frame}

%\subsection{Cosine Similarity}
\begin{frame}{Cosine Similarity}
  A standard method to compare documents is to use the cosine similarity of feature vectors:

  \[ \text{similarity}_{ij} = \cos\theta_{ij} = \dfrac{\langle x_i, x_j\rangle}{\|x_i\| \|x_j\|}. \] 

  Which features are ``turned on'' together?\\

  Using only this gets about $70\%$ on our development set.
\end{frame}


% Looks like adding the proper noun features isn't helping much here.
% this is also reflected by removing the NNP features entirely; we still score ~70% on the dev set
% The questions are feature dense (e.g. just have to find ``Birch-Murnaghan''), but we're not using this fully
% Not many links on Bulk modulus page.  If we were better at NLP, we might try to search Wikipedia for ``Birch-Murnaghan'' or ``Cauchys Number''.  That seems hard...
\begin{frame}{Cosine Similarity}
   Example: ``The Birch-Murnaghan equation of state takes volume and this parameter at zero pressure as its arguments, and Cauchys Number in compressible flows is inversely proportional to this quantity.''\\[1em]
   {\color{green}Correct}: Bulk modulus\\
   {\color{red}Our Answer}: Hamiltonian (quantum mechanics)\\[1em]

   ``equation'' and ``state'' occur very frequently in the Hamiltonian page (long), but infrequently in Bulk modulus page (short).\\
\end{frame}

%--------------------------------------------------------------------------------------------------

%\subsection{NLP to Eliminate Answers}
\begin{frame}{NLP to Eliminate Answers}
   \begin{itemize}
      \item Antibodies that localize to this organelle are associated with Sjogren\'s ("SHOW-grenz") syndrome, and a deficiency of a phosphatase localized to it causes Lowe\'s syndrome.

	(a) Golgi Apparatus (b) Apoptosis (c) Peroxisome (d) Collagen

      \item This phylum's members have a pair of pouches that split from the archenteron, with each pouch constricting into three portions. 

	(a) Xylem (b) Echinoderm (c) Sponge (d) Mitochondrian
   \end{itemize}
\end{frame}

%---------------------------------------------------------------------------------------------------

%\subsection{NLP to Eliminate Answers}
\begin{frame}{NLP to Eliminate Answers}
   \begin{itemize}
      \item Antibodies that localize to {\color{red}this} {\color{green}organelle} are associated with Sjogren\'s ("SHOW-grenz") syndrome, and a deficiency of a phosphatase localized to it causes Lowe\'s syndrome.

	(a) {\color{green}Golgi Apparatus} (b) Apoptosis (c) {\color{green}Peroxisome} (d) Collagen

      \item {\color{red}This} {\color{green}phylum's} members have a pair of pouches that split from the archenteron, with each pouch constricting into three portions. 

	(a) Xylem (b) {\color{green}Echinoderm} (c) Sponge (d) Mitochondrian
   \end{itemize}
\end{frame}

%--------------------------------------------------------------------------------------------------------------------------

%\subsection{NLP to Eliminate Answers}
\begin{frame}{NLP to Eliminate Answers}
   \begin{itemize}
      \item NLTK part-of-speech tagger to determine answer category
      \item Use Wikipedia data along with the PyGoogle module to gain insight into the answer-category relationship
      \item Eliminate options with a very low false positive rate
   \end{itemize}
\end{frame}


%----------------------------------------------------------------------------------------------------------------------------------

% Surprisingly, using LDA gets the Bulk modulus v. Hamiltonian question right. (using 75%-25% split).
%\subsection{Topic Model}
\begin{frame}{Topic Model}
   \begin{itemize}
      \item We also tried to use a topic model to categorize the Wiki pages.
      \item Doesn't work well on its own, but helps other methods a little bit.
      \item Possibly just picking the low-hanging fruit.
   \end{itemize}
\end{frame}

%----------------------------------------------------------------------------------------------------------------------------------

%\subsection{Features}
\begin{frame}{Features}
    Each Wiki page contains:
        \[
            \begin{array}{llllll}
                [ \texttt{links}, & \texttt{title}, & \texttt{summary}, & \texttt{content}, & \texttt{sections}, & \texttt{categories}]
            \end{array}\]
     Lengths of each computed and added as predictive data: \\

        \begin{itemize}
            \item {\color{blue}Magnetic field}
	\begin{itemize}
		\item \texttt{Title} $\longrightarrow$ string length of {\color{red}$14$}
		\item \texttt{Links} $\longrightarrow$ string length of {\color{red}$0$}
		\item \texttt{Summary} $\longrightarrow$ string length of {\color{red}$1924$}
		\item \texttt{Content} $\longrightarrow$ string length of {\color{red}$85336$}
		\item \texttt{Categories} $\longrightarrow$ string length of {\color{red}$0$}
	\end{itemize}
        \end{itemize}
\end{frame}

%----------------------------------------------------------------------------------------------------------------------------------

\section{Optimization}

%\subsection{Logistic Regression}
\begin{frame}{Logistic Regression}
    \begin{itemize}
        \item Optimized combination of methods via Logistic Regression.
            \begin{itemize}
                \item Main regression parameters (largest):
		\begin{itemize}
		    \item Cosine Similarity ({\color{green}$+$})
		    \item Length of Answer Choice ({\color{red}$-$})
		    \item Topic Model ({\color{green}$+$})
		\end{itemize}
	    \item Other regression parameters (less significant):
		\begin{itemize}
		    \item Lengths of Wiki Links ({\color{red}$-$})
		    \item Lengths of Wiki Summary ({\color{green}$+$})
		    \item Lengths of Wiki Content ({\color{red}$-$})
		    \item Lengths of Wiki Categories ({\color{green}$+$})
		\end{itemize}
            \end{itemize}
    \end{itemize}
\end{frame}

%----------------------------------------------------------------------------------------------------------------------------------

%\subsection{Strong Ensemble Boosting}
\begin{frame}{Strong Ensemble Boosting}
   \begin{itemize}
      \item Combine best learners together
      \item Use text based log-reg with wiki pages
      \item Use topic model and cosine similarity
      \item Normalize, and weight each learner
   \end{itemize}
\end{frame}

%----------------------------------------------------------------------------------------------------------------------------------

\section{Conclusion}
%\subsection{Results}
\begin{frame}{Conclusion: Results}
Logistic Regression Baseline: {\color{red}$62\%$}
    \begin{itemize}
        \item Additional Training Data: {\color{blue}$72\%$} cross-val, {\color{blue}$65\%$} on Kaggle
        \item Cosine Similarity: {\color{blue}$\sim 70\%$}
	\item NLP Answer Rejection using Wiki data only  {\color{blue}$\sim 40\%$} (Not included in strong ensemble boosting)
        \item Topic Model: {\color{blue}$\sim 55\%$}
        \item Logistic Regression using Similarity, Topic Model, and Length Features: {\color{blue}$78.6\%$}
    \end{itemize}
Strong Ensemble Boosting: {\color{green}$79.3\%$}

\end{frame}

\end{document}
